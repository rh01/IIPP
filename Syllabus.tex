\documentclass[a4paper,10pt]{article}

\usepackage[utf8]{inputenc}
\usepackage[encapsulated]{CJK}
\usepackage{setspace}
\usepackage{amsmath}
\usepackage{indentfirst}
\usepackage{soul,ulem}
\usepackage{color}
\usepackage[colorlinks,linkcolor=blue,anchorcolor=black,citecolor=black]{hyperref}
\usepackage{pdflscape}
\usepackage[landscape]{geometry}

\setlength{\parindent}{2em}

\title{An Introduction to Interactive Programming in Python}
%\author{Candler(申恒恒)}
\date{}
\begin{document}
\begin{CJK}{UTF8}{gkai}

\maketitle


\par This class consists of two parts that are five and four weeks long, respectively. For most weeks, you will watch two sets of videos (part a and part b) and then complete one quiz for each set. The main task for each week is to complete a mini-project that is due along with the quizzes early Sunday morning. You will then be asked to assess your peer's mini-projects on the following Sunday-Wednesday.

\section*{Syllabus}

\begin{landscape}

\begin{tabular}{cccc}

\hline
Part& Week& Topics& Mini-project\\
\hline

1&	0&	Statements, expressions, variables&	---\\
1&	1&	Functions, logic, conditionals&	"Rock-Paper-Scissors-Lizard-Spock" game\\
1&	2&	Event-driven programming, local and global variables, buttons and input fields&	"Guess the Number" game\\
1&	3&	The canvas, static drawing, timers, interactive drawing&	Stopwatch: The Game\\
1&	4&	Lists, keyboard input, motion, positional/velocity control&	"Pong" game\\
2&	5&	Mouse input, more lists, dictionaries, images&	"Memory" game\\
2&	6&	Classes, tiled images&	"Blackjack" game\\
2&	7&	Acceleration and friction, spaceship class, sprite class, sound	Spaceship from "RiceRocks" game\\
2&	8&	Sets, groups of sprites, collisions, sprite animation&	Full "RiceRocks" game\\

\hline

\end{tabular}

\end{landscape}
\end{CJK}
\end{document}